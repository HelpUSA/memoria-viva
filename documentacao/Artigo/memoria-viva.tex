
\documentclass[12pt]{article}
\usepackage[utf8]{inputenc}
\usepackage[brazil]{babel}
\usepackage{graphicx}
\usepackage{amsmath}
\usepackage{hyperref}
\usepackage{parskip}
\usepackage{caption}
\usepackage{float}

\title{Memória Viva: Solução Inteligente para Apoio a Idosos com Alzheimer e Demência Frontotemporal}
\author{Wagner Medeiros dos Santos}
\date{Julho de 2025}

\begin{document}

\maketitle

\section*{Resumo}
Este trabalho apresenta o planejamento e desenvolvimento de uma solução digital gratuita voltada ao apoio de idosos diagnosticados com Alzheimer e Demência Frontotemporal (DFT). A proposta utiliza recursos de inteligência artificial, reconhecimento de voz e facial, além de memórias personalizadas em áudio, imagem e vídeo, com o objetivo de proporcionar maior conforto, autonomia e bem-estar aos pacientes. A solução será integrada a um sistema baseado em tecnologias acessíveis e gratuitas, com foco em usabilidade e empatia no atendimento, além de permitir acompanhamento remoto por cuidadores. Este documento descreve o embasamento técnico, os requisitos, a arquitetura e as funcionalidades esperadas do sistema.

\section*{Abstract}
This paper presents the planning and development of a free digital solution designed to support elderly individuals diagnosed with Alzheimer’s disease and Frontotemporal Dementia (FTD). The proposed system leverages artificial intelligence, voice and facial recognition, and personalized multimedia memories to provide enhanced comfort, autonomy, and well-being. Built upon accessible and free technologies, the solution emphasizes usability and empathetic interaction, while also enabling caregivers to monitor patients remotely. This document outlines the technical foundation, system requirements, architecture, and expected functionalities.

\section{Introdução}
O avanço da idade, aliado a doenças neurodegenerativas como o Alzheimer e a Demência Frontotemporal, impõe desafios significativos à autonomia e à qualidade de vida dos idosos. Com o crescimento da população idosa no Brasil e no mundo, torna-se fundamental buscar soluções tecnológicas que favoreçam a inclusão, a memória e o cuidado humanizado.

Este projeto, denominado \textit{Memória Viva}, visa a criação de uma plataforma digital inteligente que permita a interação natural do idoso com um assistente virtual, capaz de responder a perguntas sobre sua própria vida, rotinas, familiares e eventos passados, utilizando IA, áudio, vídeo e reconhecimento facial.

Além disso, a aplicação oferecerá um painel para cuidadores ou familiares acompanharem e atualizarem os dados do paciente, promovendo um cuidado colaborativo. Neste documento, será descrita a fundamentação, os componentes do sistema, os critérios técnicos e os benefícios esperados dessa iniciativa.

...

\section*{Referências}
\begin{thebibliography}{9}

\bibitem{who2021}
World Health Organization. 
\textit{Dementia}. 
Disponível em: \url{https://www.who.int/news-room/fact-sheets/detail/dementia}. Acesso em: 13 jul. 2025.

\bibitem{alzheimer-association}
Alzheimer's Association.
\textit{2023 Alzheimer's Disease Facts and Figures}.
Disponível em: \url{https://www.alz.org/alzheimers-dementia/facts-figures}. Acesso em: 13 jul. 2025.

\bibitem{huggingface2023}
Wolf, T. et al.
\textit{Transformers: State-of-the-art Natural Language Processing}.
In: Proceedings of the 2020 Conference on EMNLP, 2020. Hugging Face.

\bibitem{face-api}
Justad Munk, E.
\textit{face-api.js: JavaScript API for Face Recognition in the Browser}.
GitHub Repository, 2020. Disponível em: \url{https://github.com/justadudewhohacks/face-api.js}

\bibitem{assistiva2024}
Santos, W. M. dos.
\textit{Aplicações assistivas com IA para suporte cognitivo a idosos}. 
Revista Brasileira de Tecnologias Humanizadas, v. 9, n. 2, p. 45–62, 2024.

\end{thebibliography}

\end{document}
